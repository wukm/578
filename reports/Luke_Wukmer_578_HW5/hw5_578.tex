% MATH 578 HW4
% LUKE WUKMER

\documentclass[10pt]{article}

% note: some of these are extremely useful and i don't remember why :o
%\usepackage{savetrees} % disable custom geometry stuff if you do this
\usepackage{titling}    % contol over title & stuff
\usepackage{amsmath, amsthm, amssymb, amsfonts}
\usepackage{amsxtra, amscd, geometry, graphicx}
\usepackage{endnotes}
\usepackage{cancel}
\usepackage{wrapfig}    %inline figs
\usepackage{bm} %allows fancy stuff like bold greek in math mode
\usepackage{alltt}
\usepackage{enumerate} %more/easier control over lists, also see enumitem
%\usepackage[all,cmtip]{xypic}
\usepackage{mathrsfs}
\usepackage{listings} % code with syntax highlighting etc
\usepackage{caption}
\usepackage[raggedright]{sidecap} % side captions
\usepackage{tabu}     % more customizable tables
%\usepackage{subfigure}
%\usepackage{subcaption}
%\usepackage[pdftex]{hyperref}
%\usepackage[dvips,bookmarks,bookmarksopen,backref,colorlinks,linkcolor={blue},citecolor={blue},urlcolor={blue}](hyperref}

\graphicspath{ {./figs/} }

\usepackage{color}


\definecolor{mygreen}{rgb}{0,0.6,0}
\definecolor{mygray}{rgb}{0.5,0.5,0.5}
\definecolor{mymauve}{rgb}{0.58,0,0.82}

\lstset{ %
basicstyle=\ttfamily,        % the size of the fonts that are used for the code
%breakatwhitespace=false,         % sets if automatic breaks should only happen at whitespace
breaklines=false,                 % sets automatic line breaking
captionpos=t,                    % sets the caption-position to bottom
commentstyle=\color{mygray},    % comment styleh
%  deletekeywords={...},            % if you want to delete keywords from the given language
%  escapeinside={\%*}{*)},          % if you want to add LaTeX within your code
% extendedchars=true,              % lets you use non-ASCII characters; for 8-bits encodings only, does not work with UTF-8
frame=single,                      % adds a frame around the code
%keepspaces=false,                 % keeps spaces in text, useful for keeping indentation of code (possibly needs columns=flexible)
% columns=flexible,
  keywordstyle=\color{blue},       % keyword style
  fontadjust=true,
  language=Python,                 % the language of the code
%  otherkeywords={*,...},           % if you want to add more keywords to the set
  numbers=left,                    % where to put the line-numbers; possible values are (none, left, right)
 numbersep=5pt,                   % how far the line-numbers are from the code
numberstyle=\tiny\color{mygray}, % the style that is used for the line-numbers
%  rulecolor=\color{black},         % if not set, the frame-color may be changed on line-breaks within not-black text (e.g. comments (green here))
showspaces=false,                % show spaces everywhere adding particular underscores; it overrides 'showstringspaces'
showstringspaces=false,          % underline spaces within strings only
%  showtabs=false,                  % show tabs within strings adding particular underscores
%  stepnumber=2,                    % the step between two line-numbers. If it's 1, each line will be numbered
  stringstyle=\color{mymauve},     % string literal style
%  tabsize=2,                      % sets default tabsize to 2 spaces
title=\lstname                   % show the filename of files included with \lstinputlisting; also try caption instead of title
}
% change up the fonts (pick one only)
%\usepackage{times}%
%\usepackage{helvet}%
%\usepackage{palatino}%
%\usepackage{bookman}%
\usepackage{dejavu}


% These are italic.
% \theoremstyle{definition}

% These are normal (i.e. not italic).
\theoremstyle{definition}

%\newtheorem{prob}{Problem}[section]
\newtheorem{prob}{Problem}
\newtheorem*{prob*}{Problem}
\newtheorem*{soln*}{Solution}
\newtheorem{soln}{Solution}


% New Commands: Common Math Symbols
\providecommand{\R}{\mathbb{R}}%
\providecommand{\N}{\mathbb{N}}%
\providecommand{\Z}{{\mathbb{Z}}}%
\providecommand{\sph}{\mathbb{S}}%
\providecommand{\Q}{\mathbb{Q}}%
\providecommand{\C}{{\mathbb{C}}}%
\providecommand{\F}{\mathbb{F}}%
\providecommand{\quat}{\mathbb{H}}%

% haha, i originally forked this template from one provided by my abstract
% algebra TA (back in 2012 or something). probably don't need most of these,
% huh. 

% New Commands: Operators
%\providecommand{\Gal}{\operatorname{Gal}}%
%\providecommand{\GL}{\operatorname{GL}}%
%\providecommand{\card}{\operatorname{card}}%
%\providecommand{\coker}{\operatorname{coker}}%
%\providecommand{\id}{\operatorname{id}}%
%\providecommand{\im}{\operatorname{im}}%
%\providecommand{\diam}{{\rm diam}}%
%\providecommand{\aut}{\operatorname{Aut}}%
%\providecommand{\inn}{\operatorname{Inn}}%
%\providecommand{\out}{{\rm Out}}%
%\providecommand{\End}{{\rm End}}%
%\providecommand{\rad}{{\rm Rad}}%
\providecommand{\rk}{{\rm rank}}%
%\providecommand{\ord}{{\rm ord}}%
%\providecommand{\comp}{{\text{ $\scriptstyle \circ$ }}}%
\providecommand{\cl}[1]{\overline{#1}}%
\providecommand{\tr}{{\sf trace}}%
\providecommand{\spn}{{\rm span}}%

\renewcommand{\tilde}[1]{\widetilde{#1}}%
%\numberwithin{equation}{section}

% i like the squiggly ones more. add as needed

\renewcommand{\Psi}{\varPsi}

\newcommand*\rfrac[2]{{}^{#1}\!/_{#2}}

% a very fancy dot product \ip{f}{g}
\newcommand\ip[2]{ \left\langle {#1} , {#2} \right\rangle }

% "s.t." for math mode
\providecommand{\st}{\text{ s.t. }}

% \norm{f} and such, super useful
\newcommand{\norm}[1]{\left\lVert#1\right\rVert}

% determinant
%\newcommand{\det}[1]{\textsf{det}\left(#1\right)}

% jacobian
\providecommand{\J}{\textsf{J}}

% this makes the spacing between lines of font a little bigger
%\newcommand{\spacing}[1]{\renewcommand{\baselinestretch}{#1}\large\normalsize}
%\spacing{1.2}

\DeclareMathOperator*{\argmin}{arg\,min}
\DeclareMathOperator*{\argmax}{arg\,max}

\newcommand*\mcol[1]{\overset{\big\uparrow}{\underset{\big\downarrow}{#1}}}

% Makes the margin size a little smaller, i gots stuff to say
\geometry{letterpaper,margin=.8in}

% titling stuff (from package titling)
\posttitle{\par\end{center}}
\setlength{\droptitle}{-.5in}
% END PREAMBLE %%%%%%%%%%%%%%%%%%%%%%%%%
%%%%%%%%%%%%%%%%%%%%%%%%%%%%%%%%%%%%%%%%


\begin{document}

\title{Math 578 HW\textsuperscript{\#}5}
\author{Luke Wukmer}
\date{Fall 2016}
\maketitle \thispagestyle{empty} % remove the page number from the first page


%%%% PROBLEM 1


    The entirety of this code is contained in the included single file \texttt{hw5.py}. 
    The main executing code that solves problems \#1-3 is contained within the (\verb;if __name__ == "__main__": ;) clause. See the \textbf{Appendix} for code for all three problems.
    \begin{prob} 
    \begin{enumerate}[\bfseries(a)]
        \item
           The system for problem 1 is generated by \texttt{p1\_system()}. PCG is
           implemented in pcg() (which is identical to the code used on the final
           project, with the exception of an additional optional parameter
           \texttt{max\_iterations}). Refer to the \textbf{Appendix}.
           
        \item Iteration counts and size of $\norm{Ax-b}$ for systems of the size
        $N$ for CG and PCG (with Jacobi preconditioning) is shown in the table below:
        
        \begin{table}[h]
        \begin{center}
       \begin{tabular}{|c|c|c|c|}
       N & method & iterations & $\norm{Ax-b}_2$ \\
       \hline
       10 & \textbf{CG} & 10 & 1.605e-15 \\
       \hline
       50 & & 62 & 1.217e-12 \\
       \hline
       100 &  & 200 & 0.001 \\
       \hline
       200 &  & 400 & 362.102 \\
       \hline
       500 &  & 1000 & 3126.407 \\
       \hline
       \hline
       10 & \textbf{PCG (Jacobi)} &10 &5.768e-16 \\
       \hline
       50 & &18 & 6.405e-12 \\
       \hline
       100 & &19 & 8.602e-13 \\
       \hline
       200 & &19 &8.842e-13 \\
       \hline
       500 & &19 &9.519e-13 \\
       \hline
        \end{tabular}
        \caption{Problem 1(b) output}
        \end{center}
        \end{table}
        
	For each of these runs, pcg was run with \verb;max_iterations=2*N; .
	Of course, \textit{theoretically} CG should converge successfully within $N$
	iterations. Notice that for larger systems CG simply does not converge--in fact,
	the residual norm increases. It's pretty easy to say in this case that CG is
	not a useful iterative method for this particular ill-conditioned system. This is compared to PCG, which begins to converge in a constant number of iterations, regardless of change to problem size $N$.
\end{enumerate}
\end{prob}
\begin{prob}
The system for problem 2 is constructed by \texttt{p2\_system()}.
\begin{enumerate}[\bfseries(a)]
\item Check inverse. Let $B$ be the matrix given by
\[
B = \frac{1}{2}\left( I + \sum_{k=1}^{\infty}\left[\frac{\alpha^k}{2^k} W^k\right]
\right)
\]
Then
\begin{align*}
	AB  &= \left(2I - \alpha W\right)B \\
	&= \frac{1}{2} \left(2I - \alpha W\right)\left( I + \sum_{k=1}^{\infty}\left[\frac{\alpha^k}{2^k} W^k\right]
	\right)\\
	&= \frac{1}{2} \left(2I - \alpha W + 2\sum_{k=1}^{\infty}\left[\frac{\alpha^k}{2^k} W^k\right]
	- \alpha W \sum_{k=1}^{\infty}\left[\frac{\alpha^k}{2^k} W^k\right]
		\right)\\
	&= I - \frac{1}{2}\alpha W
		+ \sum_{k=1}^{\infty}\left[\frac{\alpha^k}{2^k} W^k\right]
		- \sum_{k=1}^{\infty}\left[\frac{\alpha^{k+1}}{2^{k+1}} W^{k+1}\right] =  I
\end{align*}
as desired, so the infinite series $B$ is in fact the inverse of $A$.
\item
Using the first three terms of this representation of the inverse, PCG was run again. Output is given in Table \#2.
        \begin{table}[h]
        \begin{center}
       \begin{tabular}{|c|c|c|c|}

       N & method & iterations & $\norm{Ax-b}_2$ \\
       \hline
50& \textbf{CG} &26 &2.662e-15 \\ \hline
100 & & 51& 3.908e-15 \\ \hline
200& & 101 &1.072e-14\\ \hline
500&  &198& 8.682e-13\\ \hline
50& \textbf{PCG}&13&4.402e-15\\ \hline
100& &26 &4.148e-15\\ \hline
200& & 51& 6.312e-15\\ \hline
500& & 99& 8.682e-13\\ \hline

\end{tabular}\end{center}
\caption{Iteration counts for the problem $Ax=b$ in Problem 2(b).}
\end{table}
\item In the (P)CG algorithm there is exactly 1 multiplication by $A$ and 1 multiplication by $M^{-1}$ per iteration. In the case of no preconditioning (conventional CG), the $M^{-1}$ multiplication is clearly omitted. In terms of this problems system and choice of $M^{-1}$, each of these counts as one $A-$multiplication. Thus the number of A multiplications required for  CG is $N+1$, while PCG is $2(N+1)$, including the initialization step. Fr the table above see see that PCG converges in almost \textit{exactly half} as many iterations as CG, while requiring twice as much "work" for the problem in question. I guess they're equivalent for solving this particular $Ax=b$.
\end{enumerate}
\end{prob}
\begin{prob}{\textbf{(GMRES)}}
\begin{enumerate}[\bfseries(a)]
\item Code for \texttt{gmres()}, as well as \texttt{apply\_givens()} and such is contained in the \textbf{Appendix}.
\item I couldn't get the fucking thing to work.
\end{enumerate}
\end{prob}

\appendix
\section{Relevant Code}

\lstinputlisting{hw5.py}
\end{document}
